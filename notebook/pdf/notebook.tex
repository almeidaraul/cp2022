\documentclass[12pt, a4paper, twoside]{article}
\usepackage[T1]{fontenc}
\usepackage[utf8]{inputenc}
\usepackage{amssymb,amsmath}
\usepackage{comment}
\usepackage{datetime}
\usepackage[pdfusetitle]{hyperref}
\usepackage[all]{xy}
\usepackage{graphicx}
% \addtolength{\parskip}{.5\baselineskip}

\usepackage{listings} 
\usepackage{color}
\usepackage{titlesec}
%\usepackage[landscape, left=1cm, right=1cm, top=1cm, bottom=2cm]{geometry}
\usepackage[portrait, left=0.75cm, right=0.75cm, top=1cm, bottom=2cm]{geometry}

\definecolor{gray}{rgb}{0.4, 0.4, 0.4}
\setlength{\columnseprule}{0.2pt}
\setlength{\columnsep}{30pt}

\titleformat*{\section}{\Large\bfseries}
\titleformat*{\subsection}{\large\bfseries}


\lstset{
breaklines=true,
keywordstyle=\color{blue},
commentstyle=\color{gray},
basicstyle=\footnotesize,
breakatwhitespace=true,
language=C++,
%frame=single,
numbers=left,
numberstyle=\ttfamily\tiny,
numbersep=5pt,
rulecolor=\color{black},
showstringspaces=false
stringstyle=\color{blue},
tabsize=4,
basicstyle=\ttfamily\scriptsize,
literate={~} {$\sim$}{1}
}

\title{Competitive Programming Notebook}
\author{Raul Almeida}


\begin{document}
\twocolumn
\date{}
\maketitle

\tableofcontents

%%%%%%%%%%%%%%%%%%%%
% Graph
%%%%%%%%%%%%%%%%%%%%

\section{Graph}

\subsection{Prim MST}
\begin{lstlisting}
// take smallest edge that leads to vertex v
//
// Time: O(E log V)
// Space: O(V + E)
// Status: tested (UVA10048)

vector<vector<pair<int, int>> adj(M), mst(M);
vector<bool> taken(M, false);
int cost = 0;
using iii = pair<int, pair<int, int>>;
priority_queue<iii, vector<iii, greater<iii> pq;

void process(int v) {
	taken[v] = true;
	for (auto &[w, u]: adj[v])
		if (!taken[u])
			pq.push({w, {v, u}});
}

void run(int n) {
	process(0);
	while (!pq.empty()) {
		int w = pq.top().first,
				v = pq.top().second.first,
				u = pq.top().second.second;
		pq.pop();
		if (!taken[u]) {
			mst_cost += w;
			mst[u].push_back({w, v});
			mst[v].push_back({v, w});
			process(u);
		}
	}
	for (int v = 1; v < n; ++v)
		if (!taken[v]) {
			process(v);
			run(n);
		}
}
\end{lstlisting}

\subsection{Dijkstra SSSP}
\begin{lstlisting}
// Time: O((V+E) log V)
// Space: O(V^2)
//
// Status: tested (CF20C)

using ii = pair<int, int>;
const int inf = 0x3f3f3f3f;
vector<vector<ii>> adj(M);
vector<int> dist(M, inf), par(M, -1);

void dijkstra(int s) {
	dist[s] = 0;
	priority_queue<ii, vector<ii>, greater<pair<int, int>>> pq;
	pq.push(make_pair(0, s));
	while (!pq.empty()) {
		int w = pq.top().first;
		int v = pq.top().second;
		pq.pop();
		if (w > dist[v]) continue;
		for (auto &[d, u]: adj[v])
			if (dist[v] != inf && dist[v]+d < dist[u]) {
				par[u] = v;
				dist[u] = dist[v]+d;
				pq.push(make_pair(dist[u], u));
			}
	}
}
\end{lstlisting}

\subsection{Graph Check}
\begin{lstlisting}
// Check for tree edges, back edges and forward edges
//
// Usage: graphCheck(firstVertex, -1) (p stands for parent)
// Time: O(V + E)
// Space: O(V + E)

int UNVISITED = -1, EXPLORED = 0, VISITED = 1;
vector<vector<int>> adj(M);
vector<int> tin;

void graphCheck(int v, int p) { //vertex, parent
	tin[v] = EXPLORED;
	for (auto u: adj[v]) {
		if (tin[u] == UNVISITED) { //tree edge
			graphCheck(u, v);
		} else if (tin[u] == EXPLORED) {
			if (u == p)
				; //two way edge u <-> v
			else
				; //back edge v -> u
		} else if (tin[u] == VISITED) {
			; //forward/cross edge u-v
		}
	}
	tin[v] = VISITED;
}
\end{lstlisting}

\subsection{Articulations and Bridges}
\begin{lstlisting}
// Articulation point v: if you remove vertex v, then the connected component to which it belongs becomes disconnected
// Bridge u,v: if you remove the edge u->v, then you can't reach v from u
//
// Usage: dfs(source, -1)
//
// Time: O(V + E)
// Space: O(V + E)
// Status: not tested

int tk = 0;
vector<int> tin(M, -1);
vector<vector<int>> adj(M);

void dfs(int v, int p) {
	tin[v] = low[v] = tk++;
	int children = 0;
	for (auto u: adj[v]) {
		if (u == p) continue;
		else if	(tin[u] == -1) {
			++children;
			dfs(u, v);
			if (low[u] >= tin[v] && p != v)
				; //articulation point
			if (low[u] > tin[v])
				; //bridge u-v
			low[v] = min(low[v], low[u]);
		} else {
			low[v] = min(low[v], tin[u]);
		}
	}
}
\end{lstlisting}

\subsection{Euler Tour}
\begin{lstlisting}
// Find the closest neighbor that has a path back to the current vertex to build an euler tour
//
// Euler path: visits each edge once
// Tour/cycle/circuit: euler path that starts and ends at same node
//
// Undirected and has path: every vertex has even degree or two have odd degree
// Undirected and has circuit: every vertex has even degree
// Directed and has path: indeg[i]-outdeg[i] == 1 for at most one i, -1 for at most one i, 0 for all other i
// Directed and has circuit: indeg[i] == outdeg[i] for every i
//
// Usage: tour(cyc.begin(), start\_vertex)
// Time: O(E^2)
// Status: not tested
// Source: CP3 (pg. 205)

list<int> cyc;
vector<vector<int>> adj(M);
vector<vector<bool>> traversed(M, vector<bool>(M, false));

//euler tour (list for fast insertion)
void tour(list<int>::iterator i, int v) {
	for (auto u: adj[v]) {
		if (!traversed[v][u]) {
			traversed[v][u] = true;
			for (auto t: adj[u])
				if (t == v && !traversed[u][t]) {
					traversed[u][t] = true;
					break;
				}
			tour(cyc.insert(i, v), u);
		}
	}
}
\end{lstlisting}

\subsection{Kahn's topological sort}
\begin{lstlisting}
// particular order (alphabetical)
// Time: O(VE)
// Space: O(V+E)
// Status: tested (UVA11060)

vector<vector<int>> adj(M);
vector<int> sorted;

void kahn(int n) {
	vector<int> indeg(n, 0);
	vector<bool> valid(n, true);
	priority_queue<int> pq;

	for (int v = 0; v < n; ++v)
		for (auto u: adj[v])
			indeg[u]++;
	for (int v = 0; v < n; ++v)
		if (!indeg[v]) pq.push(v);

	while (!pq.empty()) {
		int v = pq.top(); pq.pop();
		sorted.push_back(v);
		valid[v] = false;
		for (auto u: adj[v])
			if (valid[u] && !(--indeg[u]))
				pq.push(u);
	}
}
\end{lstlisting}

\subsection{Max Cardinality Bipartite Matching}
\begin{lstlisting}
// Jump from free to matched edges until you've used them all
//
// Time: O(VE)
// Status: not tested
// Source: CP3 (pg. 209)

vector<vector<int>> adj(M);
vector<int> match(M, -1);
vector<bool> visited(M);

bool augment(int left) { //match one on the left with one on the right
	if (visited[left]) return false;
	visited[left] = true;
	for (auto right: adj[left])
		if (match[right] == -1 || augment(match[right])) {
			match[right] = left;
			return true;
		}
	return false;
}

//usage
//(mcbm = V iff there's at least one way to completely match both sides)
int mcbm = 0; //number of matched vertices
match.assign(M, -1);
for (int v = 0; v < ls; ++v) {//ls = size of the left set
	visited.assign(ls, false);
	mcbm += augment(v);
}
\end{lstlisting}

\subsection{Lowest Common Ancestor}
\begin{lstlisting}
// Binary lift to binary search the LCA or euler path
//
// Time: O(N log N)
// Space: O(N log N)
// Status: not tested

//--- binary lifting
int n, l = ceil(log2(n));
vector<vector<int>> adj;
int tk = 0;
vector<int> tin(n), tout(n);
vector<vector<int>> up(n, vector<int>(l+1)); // ancestor

void dfs(int v, int p) { // run dfs(root, root) to initialize
	tin[v] = ++tk;
	up[v][0] = p;
	for (int i = 1; i <= l; ++i)
		up[v][i] = up[up[v][i-1]][i-1];
	for (int u : adj[v])
		if (u != p)
			dfs(u, v);
	tout[v] = ++tk;
}

bool ancestor(int v, int u) { // v is ancestor of u
	return tin[v] <= tin[u] && tout[v] >= tout[u];
}

int lca(int v, int u) {
	if (ancestor(v, u)) return v;
	if (ancestor(u, v)) return u;
	for (int i = l; i >= 0; --i)
		if (!ancestor(up[v][i], u))
			v = up[v][i];
	return up[v][0];
}

//--- euler path
using ii = pair<int, int>;
vector<ii> t;
vector<int> idx(n);
int tk = 1;

void dfs(int v, int d) { // call with dfs(root, 0);
	for (auto u : adj[v]) {
		st.update(tk, {d, v});
		tk++;
		dfs(u, d+1);
	}
	idx[v] = tk;
	st.update(tk, {d, v});
	tk++;
}

int lca(int v, int u) {
	int l = idx[v], r = idx[u];
	return st.minquery(l, r).second; // .first is depth
}
\end{lstlisting}

\subsection{Tarjan Strongly Connected Component}
\begin{lstlisting}
// A node can reach any other node in its own SCC (DFS+stack)
//
// Usage: Tarjan(N, adj)
// Time: O(V + E)
// Space: O(V + E)
// Status: tested (UVA247, UVA11838)

vector<int> tin(M, -1), low(M, -1);
vector<vector<int> adj(M);
stack<int> S;
int tk = 0;

void dfs(int v) {
	low[v] = tin[v] = tk++;
	S.push(v);
	visited[v] = true;
	for (auto u: adj[v]) {
		if (tin[u] == -1)
			dfs(u);
		if (visited[u])
			low[v] = min(low[v], low[u]);
	}
	if (low[v] == tin[v])
		while (true) {
			int u = S.top(); S.pop(); visited[u] = false;
			if (u == v) break;
		}
}
\end{lstlisting}

\subsection{Bellman-Ford SSSP}
\begin{lstlisting}
// Time: O(VE)
// Space: O(V + E)
// Status: tested (UVA1112, UVA10449)

const int inf = 0x3f3f3f3f;
vector<vector<pair<int, int>>> adj(M);
vector<int> dist(M, inf);

void bellmanFord(int n) {
	for (int i = 0; i < n-1; ++i)
		for (int v = 0; v < n; ++v)
			for (auto &[u, w]: adj[v])
				if (dist[v] != inf)
					dist[u] = min(dist[u], dist[v]+w);
}

//check if there are negative cycles
bool cycle(int n) {
	bool ans = false;
	for (int v = 0; v < n; ++v)
		for (auto &[u, w]: v)
			ans |= dist[v] != inf && dist[u] > dist[v]+w;
}
\end{lstlisting}

\subsection{Kruskal MST}
\begin{lstlisting}
// order edges by increasing weight, then use a MUF to know if each edge is useful (if it connects two previously disconnected vertices)
//
// Min Span Subgraph: previously process fixed edges
//
// Min Span Forest: count number of sets on the MUF
//
// 2nd Best MST: run kruskal; for each chosen edge, flag it as unavailable and run it without using that edge ($O(VE)$)
//
// Minimax: max edge weight on the MST (maximin: min)
//
// Usage: Kruskal(V, E, edges) (weighted edges)
//
// Time: O(E log V)
// Space: O(V + E)
// Status: tested (UVA1174)

using iii = pair<int, pair<int, int>>; //weight, two vertices
vector<iii> edges;
UnionFind muf;

int kruskal() {
	int cost = 0;
	sort(edges.begin(), edges.end());
	for (auto a: edges) {
		int w = a.first;
		pair<int, int> e = a.second;
		if (!muf.isSameSet(e.first, e.second)) {
			cost += w;
			muf.unionSet(e.first, e.second);
		}
	}
	return cost;
}
\end{lstlisting}

\subsection{Edmond Karp MaxFlow}
\begin{lstlisting}
// Ford-Fulkerson's method with BFS = O(VE) BFS calls, O(E) per BFS
//
// Vertex weights: if vertex V has a weight, create Vin (receives all in-edges of V and has an edge to Vout) and Vout(receives an edge from Vin and has all out-edges of V); Vin-Vout has the weight from V
//
// MinCut: run EdmondKarp, S-T sets are: all V that you can reach from the source with edges of positive residual capacity and all other V
//
// MultiSource/MultiSink: create a super source with infinite capacity pointing to all sources, analogous for sinks
//
// Max Cardinality Bipartite Matching: use capacity 1 on all edges and apply the multi-source and multi-sink strategies
//
// Time: O(VE^2)
// Space: O(V + E)
// Status: tested (CSES1694, CSES1695)

vector<vector<int>> capacity(M, vector<int>(M, 0)), adj(M);
vector<pair<int, int>> mc; //mincut edges

int bfs(int s, int t, vi &par) {
	fill(all(par), -1);
	par[s] = -2;
	queue<pair<int, int>> q; q.push({s, inf});
	while (!q.empty()) {
		int v = q.front().first,
				flow = q.front().second;
		q.pop();
		for (auto u: adj[v])
			if (par[u] == -1 && capacity[v][u]) {
				par[u] = v;
				int new_flow = min(flow, capacity[v][u]);
				if (u == t) return new_flow;
				q.push({u, new_flow});
			}
	}
	return 0;
}

int maxflow(int s, int t) {
	int flow = 0;
	vi par(M);
	int new_flow;
	while ((new_flow = bfs(s, t, par))) {
		flow += new_flow;
		int v = t;
		while (v != s) {
			int p = par[v];
			capacity[p][v] -= new_flow;
			capacity[v][p] += new_flow;
			v = p;
		}
	}
	return flow;
}

void mincut(int s, int t) {
	maxflow(s, t);
	stack<int> st;
	vector<bool> visited(n, false);
	vector<pair<int, int>> ans;
	st.push(0);
	while (!st.empty()) {
		int v = st.top(); st.pop();
		if (visited[v]) continue;
		visited[v] = true;
		for (auto u: adj[v])
			if (capacity[v][u] > 0)
				st.push(u);
			else
				ans.push_back({v, u});
	}
	mc.clear();
	for (auto &[v, u] : ans)
		if (!visited[u])
			mc.push_back({v, u});
}
\end{lstlisting}

\subsection{Floyd Warshall APSP}
\begin{lstlisting}
// Also works for SSSP (V <= 400)
//
// Printing path: p[i][j] set to i (last node that appears before j on the path), then p[i][j] = p[k][j] on update.
//
// Transitive Closure: weight is boolean (init as 1 if there's an edge), update with bitwise OR
//
// Minimax/Maximin: w[i][j] = min(w[i][j], max(w[i][k], w[k][j]))
//
// Finding negative/cheapest cycle: init w[i][i] = inf; run(); any w[i][i] != inf is a cycle and the smallest is the cheapest; any w[i][i] < 0 is negative
//
// This can also be used for finding SCCs (check with transitive closure)
//
// Usage: FloydWarshall(n, edges)
// Time: O(V^3+E)
// Space: O(V^2+E)
// Status: tested (UVA821, UVA1056)

struct edge { int v, u, w; };
const int inf = 0x3f3f3f3f;
vector<vector<int>> weight(M, vector<int>(M, inf));
vector<edge> edges;

void floydWarshall(int n) {
	for (auto e: edges)
		weight[e.v][e.u] = e.w;
	for (int k = 0; k < n; ++k)
		for (int i = 0; i < n; ++i)
			for (int j = 0; j < n; ++j)
				if (max(weight[i][k], weight[k][j]) < inf)
					weight[i][j] = min(weight[i][j], weight[i][k]+weight[k][j]);
}
\end{lstlisting}


%%%%%%%%%%%%%%%%%%%%
% Math
%%%%%%%%%%%%%%%%%%%%

\section{Math}

\subsection{Sieve of Eratosthenes}
\begin{lstlisting}
// Time: O(n log log n)
// Space: O(n)
// Status: not tested

bitset<11234567> pr;
vector<int> factors(M, 0);
vector<int> primes;

void sieve(int n) {
	pr.set();
	for (int i = 2; i*i <= n; ++i)
		if (pr[i]) { //factors[i] == 0
			primes.push_back(i);
			for (int p = i*i; p <= n; p += i) {
				pr[p] = false;
				factors[p]++;
			}
		}
}

// O(1) for small n, O(sieve_size) else
bool isPrime(int n) {
	int sieve_size = 11234567;
	if (n <= sieve_size) return pr[n];
	for (auto p: primes) // only works if n <= primes.back()^2
		if (!(n%p)) return false;
	return true;
}
\end{lstlisting}

\subsection{Prime Factors w/ Optimized Trial Divisions}
\begin{lstlisting}
// Time: O(Pi(sqrt(n)))
// Space: O(n)
// Status: not tested
// Source: CP3 (pg. 238)

vector<int> primes;
vector<pair<int, int>> factors;

void pf(int n) {
	for (auto p: primes) {
		if (p*p > n) break;
		int i = 0;
		while (!(n%p)) {
			n /= p;
			i++;
		}
		factors.push_back({p, i});
	}
	if (n != 1) factors.push_back({n, 1});
}
\end{lstlisting}

\subsection{Extended Euclid for solving Linear Diophantine Equations}
\begin{lstlisting}
// Time: O(log min(a, b))
// Status: not tested
// Source: CP3 (pg. 242)

int x, y, d;
void extendedEuclid(int a, int b) {
	if (b == 0) { x = 1; y = 0; d = a; return; }
	extendedEuclid(b, a%b);
	int x1 = y;
	int y1 = x - (a/b)*y;
	x = x1;
	y = x1;
}

void solve(int a, int b, int c, int i) { //i solutions
	extendedEuclid(a, b);
	if (d%c) return;
	x *= c/d;
	y *= c/d;
	do {
		cout << x << ", " << y << '\n';
		x += b/d;
		y -= a/d;
	} while (--i);
}
\end{lstlisting}

\subsection{Floyd's algorithm cycle-finding}
\begin{lstlisting}
// Slow and fast (tortoise and hare)
//
// Time: O(V)
// Space: O(1)
// Status: not tested
// Source: CPHB (p. 156)

int findCycle(int x) {
	int a, b;
	a = succ(x);
	b = succ(succ(x));
	while (a != b) {
		a = succ(a);
		b = succ(succ(b));
	}
	a = x;
	while (a != b) {
		a = succ(a);
		b = succ(b);
	}
	int first = a; // first element in cycle
	b = succ(a);
	int length = 1;
	while (a != b) {
		b = succ(b);
		length++;
	}
}
\end{lstlisting}


%%%%%%%%%%%%%%%%%%%%
% Paradigm
%%%%%%%%%%%%%%%%%%%%

\section{Paradigm}

\subsection{Coordinate Compression}
\begin{lstlisting}
// normalize vector access; can also be done with map/set but high constant
//
// Time: O(N log N)
// Status: not tested
// Source: GEMA ICMC (YouTube)

vector<int> v, vals, cv; // all of the same size, cv = compressed v
vals = v;
sort(vals.begin(), vals.end());
vals.erase(unique(vals.begin(), vals.end()), vals.end());
for (int i = 0; i < n; ++i) {
	int idx = lower_bound(vals.begin(), vals.end(), v[i]) - vals.begin();
	cv[i] = idx;
}
\end{lstlisting}

\subsection{128 Bit Integers}
\begin{lstlisting}
// GCC extension; 2^127 -1 ~ 10^38
//
// Status: not tested
// Source: GEMA (YouTube)

// cout, cerr, etc; pode dar over/underflow
ostream& operator<<(ostream& out, __int128 x) {
    if (x == 0) return out << 0;
    string s; bool sig = x < 0; x = x < 0 ? -x : x;
    while(x > 0) s += x % 10 + '0', x /= 10;
    if (sig) s += '-';
    reverse(s.begin(), s.end());
    return out << s;
}

// cin, etc; pode dar over/underflow
istream& operator>>(istream& in, __int128& x) {
    char c, neg = 0; while(isspace(c = in.get()));
    if(!isdigit(c)) neg = (c == '-'), x = 0;
    else x = c - '0';
    while(isdigit(c = in.get())) x = (x << 3) + (x << 1) - '0' + c;
    x = neg ? -x : x; return in;
}
\end{lstlisting}

\subsection{Binary Search (but beautiful)}
\begin{lstlisting}
// binary steps
//
// Time: O(log N)
// Status: not tested
// Source: CPHB

// std
int l = 0, r = n-1;
while (l <= r) {
	int m = l+(r-l)/2;
	if (array[m] == x)
		// found
	if (array[m] > x) r = m-1;
	else l = m+1;
}

// nice
int k = 0;
for (int b = n/2; b > 0; b /= 2)
	while (k+b < n && array[k+b] <= x)
		k += b;
if (array[k] == x)
	// found
\end{lstlisting}


%%%%%%%%%%%%%%%%%%%%
% String
%%%%%%%%%%%%%%%%%%%%

\section{String}

\subsection{Prefix Function (KMP)}
\begin{lstlisting}
//
// to find ocurrences of s in t, use the string s+%+t, then look for pi[i] = s.length() on the "t side"
// Time: O(N)
// Status: not tested
// Source: CP-Algorithms

vector<int> prefix(string s) {
	int n = s.length();
	vector<int> pi(n, 0); // can be optimized if you know max prefix length
	for (int i = 1; i < n; ++i) {
		int j = pi[i-1];
		while (j > 0 && s[i] != s[j])
			j = pi[j-1];
		if (s[i] == s[j])
			j++;
		pi[i] = j;
	}
	return pi;
}
\end{lstlisting}


%%%%%%%%%%%%%%%%%%%%
% Structure
%%%%%%%%%%%%%%%%%%%%

\section{Structure}

\subsection{Merge/Disjoint Union-Find}
\begin{lstlisting}
//
// Usage: UnionFind(N);
// Time: O(AM), ackerman * num_operations
// Space: O(N), elements
// Status: tested (UVA11503)

struct UnionFind {
	int N;
	vi par, rk, count;

	UnionFind(int N) : N(N), par(N), rk(N, 0), count(N, 1) {
		rep(i, 0, N) par[i] = i;
	}

	int findSet(int i) {
		return par[i] == i ? i : (par[i] = findSet(par[i]));
	}

	int unionSet(int a, int b) {
		int x = findSet(a), y = findSet(b);
		if (x != y)
			count[x] = count[y] = (count[x]+count[y]);
		if (rk[x] < rk[y])
			par[x] = y;
		else {
			par[y] = x;
			if (rk[x] == rk[y])
				rk[x]++;
		}
		return count[x];
	}

	bool isSameSet(int i, int j) {
		return findSet(i) == findSet(j);
	}
};
\end{lstlisting}

\subsection{Bottom-Up Segment Tree}
\begin{lstlisting}
//
// Usage: SegTree(N);
//
// Source: CP Handbook
// Time: O(log N)
// Space: O(N), elements
// Status: not tested

struct SegTree {
  unsigned int n;
  vector<int> tree;

  SegTree(vector<int> v) : n(v.size()), tree(2*n) {
    for (int i = 0; i < n; ++i)
      modify(i, v[i]);
  }

  int query(int a, int b) {
    a += n, b += n;
    int ans = 0;
    while (a <= b) {
      if (a%2 == 1) ans += tree[a++];
      if (b%2 == 0) ans += tree[b--];
      a >>= 1; b >>= 1;
    }
    return ans;
  }

  void modify(int k, int x) {
    k += n;
    tree[k] += x;
    for (k /= 2; k >= 1; k /= 2)
      tree[k] = tree[k<<1] + tree[(k<<1) + 1];
  }
};
\end{lstlisting}

\subsection{Segment Tree}
\begin{lstlisting}
//
// Usage: SegTree(N)

struct SegTree {
	int N;
	vi st, A;

	SegTree(int N): N(N), st(4*n), A(n) {
		init();
	}

	void init() { build(1, 0, n-1); }

	int left(int i) { return i*2; }
	int right(int i) { return i*2+1; }

	//O(N)
	void build(int v, int tl, int tr) {
		if (tl == tr) st[v] = a[tl];
		else {
			int tm = (tl+tr)/2;
			build(left(v), tl, tm);
			build(right(v), tm+1, tr);
			st[v] = max(st[left(v)], st[right(v)]);
		}
	}

	//O(log n)
	int maxquery(int v, int tl, int tr, int l, int r) {
		if (l > r) return -1;
		if (l == tl && r == tr) return st[v];
		int tm = (tl+tr)/2;
		int q1 = maxquery(left(v), tl, tm, l, min(r, tm));
		int q2 = maxquery(right(v), tm+1, tr, max(l, tm+1), r);
		return max(q1, q2);
	}

	int maxquery(int l, int r) {
		return maxquery(1, 0, n-1, l-1, r-1);
	}

	//O(log n)
	void update(int v, int tl, int tr, int p, int new_val) {
		if (tl == tr) st[v] = new_val;
		else {
			int tm = (tl+tr)/2;
			if (p <= tm)
				update(left(v), tl, tm, p, new_val);
			else
				update(right(v), tm+1, tr, p, new_val);
			st[v] = max(st[left(v)], st[right(v)]);
		}
	}

	void update(int p, int new_val) {
		update(1, 0, n-1, p-1, new_val);
	}
};
\end{lstlisting}


%%%%%%%%%%%%%%%%%%%%
% Extra
%%%%%%%%%%%%%%%%%%%%

\section{Extra}

\subsection{Bashrc}
\begin{lstlisting}
xmodmap -e 'clear lock' -e 'keycode 66=Escape' # caps -> esc
alias e=vim

BASE_CP="/home/raul/cp2022"

alias c='g++ -Wall -Wconversion -Wfatal-errors -g -O2 -std=gnu++17 -fsanitize=undefined,address'
alias c14='g++ -Wall -Wconversion -Wfatal-errors -g -O2 -std=gnu++14 -fsanitize=undefined,address'
alias p3='pypy3 -m py_compile'

tp () {
	[ -f "$1.cpp" ] && echo "$1.cpp already exists";
	[ ! -f "$1.cpp" ] && tail -n +2 $BASE_CP/code/extra/template.cpp > $1.cpp && vim $1.cpp;
}

clip () {
	if [ -f "$1" ];
	then
		cat $1 | clip.exe;
	else
		echo "$1 not found"
	fi
}
\end{lstlisting}

\subsection{Vim}
\begin{lstlisting}
set et ts=2 sw=2 ai si cindent sta
set is tm=50 nu noeb sm "cul
sy on
\end{lstlisting}

\subsection{Generator}
\begin{lstlisting}
#include <bits/stdc++.h>
using namepsace std;

int main(int argc, char *argv[]) {
  cin.tie(0); ios_base::sync_with_stdio(0);
  if (argc < 2) {
    cout << "usage: " << argv[0] << " <seed>\n";
    exit(1);
  }
  srand(atoi(argv[1]));
  // use rand() for random value
}
\end{lstlisting}

\subsection{C++ Template}
\begin{lstlisting}
#include <bits/stdc++.h>
using namespace std;
using ll = long long;

int main() {
	ios_base::sync_with_stdio(0);
	cin.tie(0);
}
\end{lstlisting}

\subsection{Stress}
\begin{lstlisting}
for (( I=0; I < 5; I++ )); do
  ./gen $I > a.in
  ./brute < a.in > expected.txt
  ./a.out < a.in > output.txt
  if diff -u expected.txt output.txt; then : ; else
    echo "--> input:"; cat a.in
    echo "--> expected output:"; cat expected.txt
    echo "--> received output:"; cat output.txt
    break
  fi
  echo -n .
done
\end{lstlisting}

\pagebreak
\end{document}
